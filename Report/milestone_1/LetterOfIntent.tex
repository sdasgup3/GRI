\documentclass[12,twoside]{article}
\usepackage{url}
\usepackage{graphicx}
\usepackage{caption}
\usepackage{listings}
\usepackage{xcolor}
\usepackage{framed}
\lstset{language=Python, keywordstyle=\color{blue}\bfseries, }
\usepackage{amsmath}

%\newcommand{\cmnt}[1]{}
%\newcommand{\Transp}[2]{\ensuremath{Tranp(#1,#2)}}
%\newcommand{\Antloc}[2]{\ensuremath{Antloc(#1,#2)}
%\newcommand{\Xcomp}[2]{\ensuremath{Xcomp(#1,#2)}}
%\newcommand{\Eval}[2]{\ensuremath{eval(#1,#2)}}
%\newcommand{\Mod}[2]{\ensuremath{mod(#1,#2)}}

\pagestyle{myheadings}

\bibliographystyle{siam}

\addtolength{\textwidth}{1.00in}
\addtolength{\textheight}{1.00in}
\addtolength{\evensidemargin}{-1.00in}
\addtolength{\oddsidemargin}{-0.00in}
\addtolength{\topmargin}{-.50in}

\hyphenation{in-de-pen-dent}


\title{\textbf{Designing an Interpreter for a dynamic language for graph algorithms}}

\author{Sandeep Dasgupta\thanks{Electronic address:
\texttt{sdasgup3@illinois.edu}}}

\begin{document}
\begin{titlepage}
\thispagestyle{empty}
\maketitle
\pagebreak
\end{titlepage}

%\begin{flushleft}
%\textbf{\Large{Motivation}}
%\end{flushleft}

\section{Motivation}

As graphical models are increasingly being used in various
  fields like biochemistry (genomics), electrical engineering (communication
      networks and coding theory), computer science (algorithms and
        computation) and operations research (scheduling), organizational
      structures, social networking, there is a need to represent and allow
      computation on them in a convenient and efficient way. This involves (but not limited
          to)

    \begin{itemize}
      \item Designing a language which provide an convenient interface to the programmer to program those models.
        This is essential so that even for domain experts who are not coding experts can code and reason about
        their implementation.
        Ease of interface could be due to:
        \begin{itemize}
          \item Expressive power of the language representing those models.
          \item Intuitive extensibility of the language.
          \item Ability of the language to provide exploratory programming, where the user
          may experiment with different ideas (without dwelling much into the language syntax) before coming to a conclusive one.
        \end{itemize}
     \item Designed language need to be efficient in the following sense.     
        \begin{itemize}
          \item Underlying design decisions including data structures need to be carefully crafted 
            to achieve expected run-time w.r.t the input size.
          \item Implementation need to be scalable w.r.t the space/time requirements. This is 
          important because most of the graph algorithm typically work
          on huge input sizes.
        \end{itemize}
   \end{itemize}       

   To meet all above goals and most importantly exploratory programming, we decided 
     to work on a dynamically typed language to represent graphs
     and apply various computations on them.
     With this the user do not have to worry much about declaring types and can
     focus mostly on his/her experiments.

 Following is the outline of the proposed features
 of the language. These are written keeping the
 ``convenience'' aspect of the language in mind.

 \begin{itemize}
   \item As ``nodes'' and ``edges'' are the backbone of any graph structure, we proposed to have them as 
  first-class objects. Even though the user will not necessarily
  be accessing individual nodes/edges, but the option is available to them.
  \item Providing users with the option to specify attributes on nodes and edges and later use
  those attributes for computing results. For example: to find all the red colored nodes (where color is marked as an attribute of
      node) connected to a specific node.
  \item Supporting  operators to specify relationships between the nodes/edges or their groups.
  For example, the user can create sub graphs like
  \begin{align}
          department\_1 &= (N1,N2,N3) \\
          managers &= (N3,N4,N5) 
  \end{align}
  and use operators like $\cap$ to find
  nodes ``department 1 employees who are managers as well''.
  Also the language supports operators to find edges between groups of nodes. 
  This will answer interesting queries like:
  $$managers \rightarrow  (department 1 - managers) $$ could mean
  ``the number of email conversation between managers to department\_1's non managers''

  Most of these ideas
  behind choosing the operators are borrowed from ~\cite{Adar}.  
  The intuition behind these operators is that the users do not have to remember longer 
  commands. Also its very intuitive to build up complex operations using the simpler ones.

   \item Also borrowed from ~\cite{Adar}, is the option of saving state of the graph.
     This seems a useful service provided to do exploratory programming, as the user might be interested to 
     checkout the last saved state  (or a state with a any tag) or to undo all the experiments
     down to a particular state.

  \end{itemize}
     

 In this project we will also be implementing an Interpreter of the above
 mentioned language. We call it GRI (Graph Interpreter).


\vspace{1 mm}

\section{GRI Interpreter}
 
Proposed language will not allow the user to specify the types and a variable
  may point to objects of different types during run time (and hence a
      dynamically types language). In this scenario we could do a static
  compilation after a static type inference to generate sub optimal machine
  code, whose efficiency depends on how well the types are inferred.

  As opposed to that,  we are trying to build an interpreter which will
  interpret the AST spit out by the parser.  If time permits, we are planning
  to convert the the AST into LLVM byte code (after static type inferencing) to be Jitted
  by the LLVM tool-chain. 

\section{Language}

The syntax of the language is going to be an oversimplified version  of C.
  Also planning to provide a pre-compiled library of rich set of built-in functions like
  \emph{deleteVertex(graph, vertex), dfs\_iterators(vertex),
    bfs\_iterators(vertex)} which will facilitate writing complex algorithms
      conveniently and enhance the execution time of interpreting.
      Some of the ideas of the language design will be borrowed from
      ~\cite{Graphal}.

  The language will distinguish between user variables/functions and system variables/functions.
  The later will be treated immutable and an
  error will be issued if the user tried to modify this. For example, a system variable g,
  representing a graph, cannot to modified (for example, by assigning
      g = 5;).  In that sense the ``dynamically typed'' nature of the language
  will be restricted to the user variables.


\bibliography{LetterOfIntent.bib}

%\nocite{*}


\end{document}
