\documentclass[12,twoside]{article}
\usepackage{url}
\usepackage{graphicx}
\usepackage{caption}
\usepackage{listings}
\usepackage{xcolor}
\usepackage{framed}
\lstset{language=Python, keywordstyle=\color{blue}\bfseries, }
\usepackage{amsmath}

%\newcommand{\cmnt}[1]{}
%\newcommand{\Transp}[2]{\ensuremath{Tranp(#1,#2)}}
%\newcommand{\Antloc}[2]{\ensuremath{Antloc(#1,#2)}
%\newcommand{\Xcomp}[2]{\ensuremath{Xcomp(#1,#2)}}
%\newcommand{\Eval}[2]{\ensuremath{eval(#1,#2)}}
%\newcommand{\Mod}[2]{\ensuremath{mod(#1,#2)}}

\pagestyle{myheadings}

\bibliographystyle{siam}

\addtolength{\textwidth}{1.00in}
\addtolength{\textheight}{1.00in}
\addtolength{\evensidemargin}{-1.00in}
\addtolength{\oddsidemargin}{-0.00in}
\addtolength{\topmargin}{-.50in}

\hyphenation{in-de-pen-dent}


\title{\textbf{Progress Report: Designing an Interpreter for a dynamic language 
  for graph algorithms}}

\author{Sandeep Dasgupta\thanks{Electronic address:
\texttt{sdasgup3@illinois.edu}}}

\begin{document}
\begin{titlepage}
\thispagestyle{empty}
\maketitle
\pagebreak
\end{titlepage}

\section{Problem Statement}

   In this project we are planning to work on a dynamically typed language to 
     represent graphs and apply various computations on them.
 
\section{Language Syntax}
  The syntax of the language is an oversimplified version  of C, but
    without the mention of any types. The operations on incompatible types
    will be error-ed out while interpreting.

  \subsection{Progress}
  \begin{itemize}
    \item We have implemented the tokenizer using flex. Appendix A shows the 
      tokens sent to the parser routine.
    \item We are supporting syntax like \#include("filename") and \#define("PI", 
        "3.14") while doing a single pass of parsing  (i.e. both
    preprocessing of these constructs done while parsing). This is achieved by 
          using flex internal stack to manage multiple buffers.
    \item Appendix B shows the parser rules. These are borrowed from 
     ~\url{http://www.quut.com/c/ANSI-C-grammar-y.html}. The rules are compiled 
     by bison tool to generate the C parser.

    \item  We are done with generating the AST. Our AST is basically a list of 
    function definitions. Each function definition class holds name of the 
    function, a set of formal arguments and a list of body statements. These body 
    statements could be a assignment, loop-statement, function call, etc.
    The leafs of the AST could be an identifier, int, float, true, false, null, string, vertex, 
    edge or graph.
    \item Some of the key features of the parser is as follows:

    \begin{itemize}
      \item Support of C statements like  \emph{if then}, \emph{if then else}, 
        \emph{while}, \emph{for}. 

      \item Support of break, continue within loop constructs and return in 
      function body.
      As we are representing both loop-body and function body (i.e. anything 
          between ``\{'' \& ``\}'') as compound statements so we do not have to
      distinguish the two cases. The semantics of executing a break, continue 
      and return will be discussed in the semantics section.

      \item Supporting vertices, edges and graphs as first class objects 
      valuevertex, valueedge, valuegraph respectively.
      The syntax to declare a graph is {\tt\emph{g = graph();}} which will be 
      parsed as a assignment node with left Value as an identifier and right
      value as a function call. Now this function call corresponds to a built in 
      function that returns a valuegraph (which is of one the leaf nodes of AST).
      \item The first class object of vertex and edge contains a map to add
      properties. This feature is useful in various graph algorithms like in dfs 
      traversal we may use a vertex property ``visited'' to keep track of 
      vertices already explored.
    \end{itemize}
    
  \end{itemize}
  


\section{Language Semantics}
\begin{itemize}
   \item As ``nodes'' and ``edges'' are the backbone of any graph structure, we 
     proposed to have them as first-class objects. Even though the user will not 
     necessarily
  be accessing individual nodes/edges, but the option is available to them.
  \item Providing users with the option to specify attributes on nodes and edges 
  and later use
  those attributes for computing results. For example: to find all the red 
  colored nodes (where color is marked as an attribute of
      node) connected to a specific node.
\end{itemize}

\section{Interpretor Runtime}

\section{Future Work}
The following are the future work.
  \begin{itemize}
    \item 
      To support  additional operators to specify relationships between 
      the nodes/edges or their groups. The operator that we are planning to 
      support are the described next with their semantics.

    \begin{itemize}
      \item v1$\leftarrow$v2: Select all the directed edges from 
        vertices v1 to v2.
      \item v1$\rightarrow$v2: Select all the directed edges from 
      vertices v2 to v1.
      \item v1$\leftrightarrow$v2: Select all the directed edges 
      between vertices v1 and v1.
      \item v1$?$v2: Select all the edges (directed or undirected) 
      between vertices v1 and v1.
      \item S1$\cap$S2: Selects the intersection between the set of
      vertices/edges.
    \end{itemize}
      Most of these ideas behind choosing the operators are borrowed from 
        ~\cite{Adar}. The intuition behind these operators is that the users 
        do not have to remember longer commands. Also its very intuitive to 
          build up complex operations using the simpler ones.

    \item 
    To support saving of state and retrieving it back using
    routines like saveSateFromFile \& loadStaeFromFile.
    This idea is borrowed from ~\cite{Adar} and this seems a useful service 
    provided to do exploratory programming, as the user might be interested to 
    checkout the last saved state  (or a state with a any tag) or to undo all 
    the experiments down to a particular state.

  \end{itemize}

\subsection{Evaluation Strategy}
The baseline of our evaluation will be the open source ~\cite{Graphal} system.  
  The evaluation will
  cover the following two aspects of our implementation:
  \begin{itemize}
    \item The convenience and intuitive extensibility provided by our 
      programming model.
    \begin{itemize}
      \item We will be implementing a couple of well know graph algorithms in 
        both baseline and our implementation
        and use number of dynamic instructions interpreted as a measure to show 
        the conciseness of our
        representation.
    \end{itemize}
    \item Performance
    \begin{itemize}
      \item As we are planning to keep the compiled version of frequently used 
        graph routines (like dfs\_iterators(vertex), bfs\_iterators()), we are 
        expecting to achieve better performance in terms of runtime.  
        \end{itemize}
  \end{itemize}



\bibliography{ProgressReport.bib}

%\nocite{*}


\end{document}
