 The syntax of the language is an oversimplified version  of C, but
    without mention of any types. The operations on incompatible types
    will be error-ed out while interpreting.

    We have implemented the tokenizer and syntax analyzer using flex and 
    bison. We are supporting syntax like \#include("filename") and 
    \#define("PI", "3.14") while doing a single pass of parsing  (i.e.  
        Preprocessing of these constructs are done while parsing). This is 
    achieved by using flex's internal stack to manage multiple buffers.
    The grammar rules are mostly borrowed from 
    ~\cite{ANSI}. The rules are compiled 
    by bison tool to generate the C parser. 
    We are able to generate the AST corresponding to test-cases 
    confirming to the grammar rules. Our AST is basically a list of function 
    definitions. Each function definition object contains name of the function, 
    a set of formal arguments and a list of body statements. These body 
    statements could be an assignment, loop-statement, function call, etc.
    The leafs of the AST could be an identifier, int, float, true, false, null, 
    string, vertex, edge or graph.

    Some of the key features of the parser is as follows:
    \begin{itemize}
      \item Support of C statements like  {\tt if then}, {\tt if then 
        else}, {\tt while}, {\tt for}, {\tt foreach}. 

      \item Support of {\tt break}, {\tt continue} within loop-body 
      and {\tt return} in function-body.
      As we are representing both loop-body and function-body as compound 
      statements (i.e. anything between ``\{'' \& ``\}''), so we do not have to
      distinguish these two cases. But we will error-out if break is used inside 
      non-loop body. The detailed semantics of executing a {\tt break},   
{\tt continue} and {\tt return} will be discussed in the interpreter 
runtime section.

      \item Supporting graph as first class object {\tt graphnode}.
      The syntax to declare a graph is {\tt g = graph();} which will be 
      represented in AST as an assignment-node with left-node containing an 
      identifier and right
      node as a function call. Now this function call corresponds to a built in 
      function that returns a {\tt graphnode} (which is of one the leaf 
          nodes of AST) on execution.

      \item Supporting vertices and edges as first class objects which contains 
      a map to add properties. This feature is useful in various graph 
      algorithms like in dfs traversal~\ref{fig:motiv_5}, we uses a vertex property ``visited'' 
      to keep track of vertices already explored.

      \item We are supporting composite data-structures like array, struct and 
      set and iterators on them. Figure~\ref{fig:language_1} represents a code 
      snippet representing some of the operation on these data-structures.

      \item The language semantics will be same as that of C as we are using a subset of it.`

      \item All variables are defined as local and are valid only it the scope of the current function (function not block).

      \item The language specify no constructs for variable declaration and type 
      specification. The interpreter uses some inner data types (like {\tt 
          null, Bool, int, float, string, array, struct, set, graph, vertex, 
          edge})  which can be dynamically changed with assign command.

    \end{itemize}
    \begin{figure}
\begin{center}
      {\small \tt
        \begin{tabular}[b]{rl}
          &function main(argv) main() \{ \\
	  {\em \scriptsize S1:}&  \quad  arr = array(5);\\
	  {\em \scriptsize S2:}&  \quad  i = 0;\\
	  {\em \scriptsize S3:}&  \quad  \ldots\\
	  {\em \scriptsize S4:}&  \quad  foreach(var ; arr)\\
	  {\em \scriptsize S5:}&  \quad \quad 	var = i++;\\
	  {\em \scriptsize S6:}&  \quad  \ldots\\
	  {\em \scriptsize S7:}&  \quad  println("--- array items ---");\\
	  {\em \scriptsize S8:}&  \quad  foreach(var ; arr)\\
	  {\em \scriptsize S9:}&  \quad \quad  	println(var);\\
	  {\em \scriptsize S10:}&  \quad  \ldots\\
	  {\em \scriptsize S11:}&  \quad  st = struct();\\
	  {\em \scriptsize S12:}&  \quad  st.number = 42;\\
	  {\em \scriptsize S13:}&  \quad  st.pi = 3.14;\\
	  {\em \scriptsize S14:}&  \quad  st.str = "bagr";\\
	  {\em \scriptsize S15:}&  \quad  \ldots\\
	  {\em \scriptsize S16:}&  \quad  println("--- struct items ---");\\
	  {\em \scriptsize S17:}&  \quad  foreach(var ; st)\\
	  {\em \scriptsize S18:}&  \quad \quad  	println(var);\\
	  {\em \scriptsize S19:}&  \quad  println("--- struct items using iterator ---");\\
	  {\em \scriptsize S20:}&  \quad  it = st.iterator();\\
	  {\em \scriptsize S21:}&  \quad  while(it.hasNext())\\
	  {\em \scriptsize S22:}&  \quad  \quad`	println(it.next());\\
	  {\em \scriptsize S23:}&  \quad  \ldots\\
	  {\em \scriptsize S24:}&  \quad  g = graph();\\
	  {\em \scriptsize S25:}&  \quad  v1 = g.generateVertex();\\
	  {\em \scriptsize S26:}&  \quad  v2 = g.generateVertex();\\
	  {\em \scriptsize S27:}&  \quad  v3 = g.generateVertex();\\
	  {\em \scriptsize S28:}&  \quad  e1 = g.generateEdge(v1, v2);\\
	  {\em \scriptsize S29:}&  \quad  e2 = g.generateEdge(v2, v3);\\
	  {\em \scriptsize S30:}&  \quad  \ldots\\
	  {\em \scriptsize S31:}&  \quad  v1.color = "red";\\
	  {\em \scriptsize S32:}&  \quad  v2.color = "green";\\
	  {\em \scriptsize S33:}&  \quad  v3.color = "blue";\\
	  {\em \scriptsize S34:}&  \quad  e1.value = 0.5;\\
	  {\em \scriptsize S35:}&  \quad  e2.value = 0.4;\\
	  {\em \scriptsize S36:}&  \quad  \ldots\\
	  {\em \scriptsize S37:}&  \quad  println("--- vertex set ---");\\
	  {\em \scriptsize S38:}&  \quad  foreach(var ; g.getVertices())\\
	  {\em \scriptsize S39:}&  \quad  \quad	println("" + var + ": " + var.color);\\
	  {\em \scriptsize S40:}&  \quad  \ldots\\
	  {\em \scriptsize S41:}&  \quad  println("--- edge set ---");\\
	  {\em \scriptsize S42:}&  \quad  foreach(var ; g.getEdges())\\
	  {\em \scriptsize S43:}&  \quad  \quad	println("" + var + ": " + var.value);\\
          &\}
        \end{tabular}
      }
\end{center}
  \caption{Example code snippet to show operatons on array, set and struct}
  \label{fig:language_1}
\end{figure}

  
