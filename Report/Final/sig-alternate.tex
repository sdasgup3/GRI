\documentclass[letterpaper]{sig-alternate}
\special{papersize=8.5in,11in}
\usepackage{url}
\newtheorem{example}{Example}
\newtheorem{definition}{Definition}
%%%%%% Non re-Com packages
\usepackage{pst-node}
\usepackage{pst-rel-points}
%%%%%%%%%%%%%%%%%%%%%%%%%%%%%%%%%%%%%%%%
%%%%%%% New Commands %%%%%%%%%%%%%%%%%%
%%%%%%%%%%%%%%%%%%%%%%%%%%%%%%%%%%%%%%%%

\begin{document}

\title{GRI: Interpreter of a dynamic language for GRaph algorithms}

\numberofauthors{1} 
% reasons) and the remaining two appear in the \additionalauthors section.
%
\author{
%\alignauthor Sandeep Dasgupta\\
\alignauthor Sandeep Dasgupta\\
       \affaddr{University Of Illinois at Urbana Champaign.}\\
       \email{sdasgup3@illinois.edu}
}
\date{}

\maketitle
\begin{abstract}
we decided 
     to work on a dynamically typed language to represent graphs
     and apply various computations on them.
     With a dynamically typed language the user do not have to worry much about declaring types and can
     focus mostly on his/her experiments.

\end{abstract}

\keywords{Graph Algorithm, dynamically typed language, interpreter} 

%%%%%%%%%%%%%%%%%%%%%%%%%%%%%%%%%%%%%%%%%%%%%%%%%%%%%
\section{Introduction}
%%%%%%%%%%%%%%%%%%%%%%%%%%%%%%%%%%%%%%%%%%%%%%%%%%%%%
As graphical models are increasingly being used in various
  fields like biochemistry (genomics), electrical engineering (communication
      networks and coding theory), computer science (algorithms and
        computation) and operations research (scheduling), organizational
      structures, social networking, there is a need to represent and allow
      computation on them in a convenient and efficient way. This involves (but not limited
          to)

    \begin{itemize}
      \item Designing a language which provide an convenient interface to the programmer to program those models.
        This is essential so that even for domain experts who are not coding experts can code and reason about
        their implementation.
        Ease of interface could be due to:
        \begin{itemize}
          \item Expressive power of the language representing those models.
          \item Intuitive extensibility of the language.
          \item Ability of the language to provide exploratory programming, where the user
          may experiment with different ideas (without dwelling much into the language syntax) before coming to a conclusive one.
        \end{itemize}
     \item Designed language need to be efficient in the following sense.     
        \begin{itemize}
          \item Underlying design decisions including data structures need to be carefully crafted 
            to achieve expected run-time w.r.t the input size.
          \item Implementation need to be scalable w.r.t the space/time requirements. This is 
          important because most of the graph algorithm typically work
          on huge input sizes.
        \end{itemize}
   \end{itemize}       


%%%%%%%%%%%%%%%%%%%%%%%%%%%%%%%%%%%%%%%%%%%%%%%%%%%%%
\section{Related Work}\label{sec:bgrel}
%%%%%%%%%%%%%%%%%%%%%%%%%%%%%%%%%%%%%%%%%%%%%%%%%%%%%

%%%%%%%%%%%%%%%%%%%%%%%%%%%%%%%%%%%%%%%%%%%%%%%%%%%%%
\section{A Motivating Example}\label{sec:motiv}
%%%%%%%%%%%%%%%%%%%%%%%%%%%%%%%%%%%%%%%%%%%%%%%%%%%%%


%%%%%%%%%%%%%%%%%%%%%%%%%%%%%%%%%%%%%%%%%%%%%%%%%%%%%
\section{Definitions and Notations} \label{sec:Formal_Definitions}
%%%%%%%%%%%%%%%%%%%%%%%%%%%%%%%%%%%%%%%%%%%%%%%%%%%%%


%%%%%%%%%%%%%%%%%%%%%%%%%%%%%%%%%%%%%%%%%%%%%%%%%%%%%
\subsection{Interprocedural Analysis}
\label{Interprocedural_Analysis}
%%%%%%%%%%%%%%%%%%%%%%%%%%%%%%%%%%%%%%%%%%%%%%%%%%%%%



\bibliographystyle{abbrv}
%\bibliography{sigproc}  

\end{document}
