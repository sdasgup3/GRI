As graphical models are increasingly being used in various fields like
biochemistry (genomics), electrical engineering (communication networks and
    coding theory), computer science (algorithms and computation) and
operations research (scheduling), organizational structures, social networking,
           there is a need to represent and allow computation on them in a
convenient and efficient way. This involves (but not limited to)

    \begin{itemize} 
    \item Designing a language which provide an convenient
    interface to the programmer to program those models.  This is essential so
    that even for domain experts who are not coding experts can code and reason
    about their implementation.  Ease of interface could be due to:
      \begin{itemize} 
        \item Expressive power of the language representing those
    models.  
        \item Intuitive extensibility of the language.  
        \item Ability of the language to provide exploratory programming, 
        where the user may
    experiment with different ideas (without dwelling much into the language
        syntax) before coming to a conclusive one.  
      \end{itemize} 
    \item
    Designed language need to be efficient in the following sense.
      \begin{itemize} 
        \item Underlying design decisions including data structures
    need to be carefully crafted to achieve expected run-time w.r.t the input
    size.  
        \item Implementation need to be scalable w.r.t the space/time
    requirements. This is important because most of the graph algorithm
    typically work on huge input sizes.  
      \end{itemize} 
    \end{itemize}       


