Our work in mostly inspired by the line of work by GUESS ~\cite{Adar} and
Graphal ~\cite{Graphal}.

GUESS, a novel system for graph
exploration that combines an interpreted language with a
graphical front end that allows researchers to rapidly prototype
and deploy new visualizations. GUESS also contains a novel,
interactive interpreter that connects the language and interface in
a way that facilities exploratory visualization tasks. They used a domain 
specific embedded language
which provides all the advantages of Python with new graph
specific operators, primitives, and shortcuts.

Graphal is an interpreter of a programming language that is mainly oriented to
graph algorithms. There is a command line interpreter and a graphical
integrated development environment. The IDE contains text editor for
programmers, compilation and script output, advanced debugger and visualization
window. The progress of the interpreted and debugged graph algorithm can be
displayed in 3D scene.

Our language design is inspired by the above  two work. But we additionally provided
a number of built-in functions for supporting some basic computations on graphs. This not only help us getting
convenient short hand notations to achieve those basic computation, but also we gain on 
performance due the fact that those basic tasks are now available in compiled 
version.
