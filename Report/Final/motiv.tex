In this section we will provide some insights of designed language
using some motivating examples.

\begin{example}{\rm
Figure\ref{fig:motiv_1}, shows a directed graph where the nodes represent the 
  employees
and the edges between them represents the email conversation from one employee 
to other. For example, {\tt Node 0} represents an employee in {\tt DPT\_x} who 
is a Manager as well (this is attributed by the {\tt MGR} tag). Similarly, {\tt 
  Node 3} represents a {\tt DPT\_x} employee. The edge between {\tt Node 3} and 
  {\tt Node 0} represents the email conversation from employee {\tt Node 3} to 
employee {\tt Node 0}.

\begin{figure}
  \begin{center}
    \fig{Figs/0}
  \end{center}
  \caption{An example graph}
  \label{fig:motiv_1}
\end{figure}

  Now let us first talk about the ways to represent this graph in our 
    implementation. One way is as 
    shown in ~\ref{fig:motiv_2}. 

\begin{figure}
\begin{center}
      {\small \tt
        \begin{tabular}[b]{rl}
          &function main(argv) main() \{ \\
          {\em \scriptsize S1:}& \quad g = graph(); \\
          {\em \scriptsize S2:}& \quad \ldots        \\
          {\em \scriptsize S3:}& \quad g.setDirected(true); \\
          {\em \scriptsize S4:}& \quad \ldots \\ {\em \scriptsize S5:}& \quad v0  
          = g.createVertex(); \\
          {\em \scriptsize S6:}& \quad v0.\_\_id     = 0; \\
          {\em \scriptsize S7:}& \quad v0.\_\_DPT\_x  = 1; \\
          {\em \scriptsize S8:}& \quad v0.\_\_DPT\_y  = 0; \\
          {\em \scriptsize S9:}& \quad v0.\_\_MGR    = 1; \\
          {\em \scriptsize S10:}& \quad \ldots \\ {\em \scriptsize S11:}& \quad 
          v3  = g.createVertex(); \\
          {\em \scriptsize S12:}& \quad v3.\_\_id     = 3; \\
          {\em \scriptsize S13:}& \quad v3.\_\_DPT\_x  = 1; \\
          {\em \scriptsize S14:}& \quad v3.\_\_DPT\_y  = 0; \\
          {\em \scriptsize S15:}& \quad v3.\_\_MGR    = 0; \\
          {\em \scriptsize S16:}& \quad \ldots \\ {\em \scriptsize S17:}& \quad 
          g.createEdge(v0,v3); \\
          &\}
        \end{tabular}
      }
\end{center}
  \caption{Example code snippet to create a graph}
  \label{fig:motiv_2}
\end{figure}
  
  
  
  At line {\tt S1}, a new graph variable is created. {\tt S3} sets that
  the graph is directed. {\tt S5} creates a node {\tt Node 0} of this graph. As 
  nodes and edges are the basic building block of a graph we have made them the 
  first class objects which allow users to access them directly. Lines {\tt S6} 
  - {\tt S9} sets various properties of the vertex. For example, it says that 
    the vertex has {\tt \_\_id = 0 }, it belongs to  {\tt \_\_DPT\_x}, does not 
    belongs to {\tt \_\_DPT\_y} and its {\tt \_\_MGR} property is true. Similar 
    properties are set for vertex node with {\tt \_\_id = 3} (Lines {\tt S11 - 
        S15}).
  And finally a directed  edge between them is created at line {\tt S17}. 

  One thing to note here is that the methods like {\tt graph()}, {\tt 
    setDirected()}, {\tt createVertex()} and {\tt createEdge()} are all built-in 
    compiled functions. 

  \begin{figure}
\begin{center}
      {\small \tt
        \begin{tabular}[b]{rl}
          &function main(argv) main() \{ \\
          {\em \scriptsize S1:}&  \quad  g = graph();\\
          {\em \scriptsize S2:}&  \quad  g.loadFromFile(argv[0]);\\
          {\em \scriptsize S3:}&  \quad  
          displayAdjMatrix(g.getAdjacencyMatrix(), g.getVertices()); \\
          {\em \scriptsize S4:}&  \quad  \ldots \\
          {\em \scriptsize S5:}&  \quad  dpt\_x\_employee  = 
          g.getVertexSetWithProperty("\_\_DPT\_x", 1.0); \\
          {\em \scriptsize S6:}&  \quad  mgr\_employee    = 
          g.getVertexSetWithProperty("\_\_MGR", 1.0); \\
          {\em \scriptsize S7:}&  \quad  \ldots \\
          {\em \scriptsize S8:}&  \quad /* Set of \_\_DPT\_x employees who are 
                                           \_\_MGR as well */  \\
          {\em \scriptsize S9:}&  \quad dpt\_x\_AND\_mgr = 
          dpt\_x\_employee.intersection(mgr\_employee); \\
          {\em \scriptsize S10:}&  \quad  \ldots \\
          {\em \scriptsize S11:}&  \quad println("Set of \_\_DPT\_x employees 
          who are \_\_MGR as well");  \\
          {\em \scriptsize S12:}&  \quad foreach(employee; dpt\_x\_AND\_mgr) \{ 
          \\
          {\em \scriptsize S13:}&  \quad \quad println(" " + employee.\_\_id + " 
          "); \\
          {\em \scriptsize S14:}&  \quad \}   \\
          {\em \scriptsize S15:}&  \quad  \ldots \\
          {\em \scriptsize S16:}&  \quad /* Set of \_\_DPT\_x employees who are 
                                            NOT  \_\_MGR */ \\
          {\em \scriptsize S17:}&  \quad dpt\_x\_MINUS\_mgr = 
          dpt\_x\_employee.difference(mgr\_employee); \\
          {\em \scriptsize S18:}&  \quad    \ldots \\
          {\em \scriptsize S19:}&  \quad it = dpt\_x\_MINUS\_mgr.iterator(); \\
          {\em \scriptsize S20:}&  \quad    \ldots \\
          {\em \scriptsize S21:}&  \quad println("Set of \_\_DPT\_x employees 
          who are NOT  \_\_MGR"); \\
          {\em \scriptsize S22:}&  \quad while(it.hasNext()) \{\\
          {\em \scriptsize S23:}&  \quad \quad employee = it.next();\\
          {\em \scriptsize S24:}&  \quad \quad println(" " + employee.\_\_id + " 
          ");\\
          {\em \scriptsize S25:}&  \quad \}  \\
          {\em \scriptsize S26:}&  \quad  \ldots \\
          {\em \scriptsize S27:}&  \quad // Email Exchanges from MGRs to non-MGR 
          DPT\_x employee \\
          {\em \scriptsize S28:}&  \quad emailExchanges = mgr\_employee -> 
          dpt\_x\_MINUS\_mgr;\\
          {\em \scriptsize S29:}&  \quad  \ldots \\
          {\em \scriptsize S30:}&  \quad // Email Exchanges from non-MGR DPT\_x 
          employee to MGRs*/\\
          {\em \scriptsize S31:}&  \quad emailExchanges = mgr\_employee <- 
          dpt\_x\_MINUS\_mgr;\\
          {\em \scriptsize S32:}&  \quad  \ldots \\
          {\em \scriptsize S33:}&  \quad // Email Exchanges between non-MGR 
          DPT\_x employee and MGRs*/\\
          {\em \scriptsize S34:}&  \quad emailExchanges = mgr\_employee <-> 
          dpt\_x\_MINUS\_mgr;\\
          {\em \scriptsize S35:}&  \quad  \ldots \\
          &\}
        \end{tabular}
      }
\end{center}
  \caption{Example code snippet to create a graph}
  \label{fig:motiv_3}
\end{figure}

  \begin{figure}
  \begin{center}
    \fig{Figs/2.1.eps}
  \end{center}
  \caption{The input file used for graph creation}
  \label{fig:motiv_4}
\end{figure}


  Figure~\ref{fig:motiv_3} shows another way to represent graph.  
  It uses an input file to be fed to
  the interpreted program. The format of the input file to represent the graph 
  at Figure~\ref{fig:motiv_1} is shown in 
  Figure~\ref{fig:motiv_4}. Such an input file is read from the command line 
  arguments and 
  used to populate a graph variable as 
  shown in line {\tt S2} of Figure~\ref{fig:motiv_3}. At {\tt S3}, 
       we can see two built-in functions {\tt getAdjacencyMatrix} \& {\tt 
         getVertices} on graph
  which respectively gives an array of array representing the adjacency matrix 
  representation of the graph and a set of vertices in the graph. At the same 
  line, we have a method call {\tt displayAdjMatrix}, the definition of which is not 
    shown for brevity. Now lets try to solve certain queries using the graph at 
    Figure~\ref{fig:motiv_1}.

  In case we want to get all the nodes in the graph who are {\tt \_\_DPT\_x} 
  employees, then
  the query to get all those nodes is shown at line {\tt S5}. Similarly, the query at 
  line {\tt S6}, gives
  the nodes for which the property {\tt \_\_MGR} is set to true.  One thing to 
  note here that the return value of both the queries are sets which are 
  amenable to set operations.

  For example, in case we want to get all the employees who are both department 
  {\tt \_\_DPT\_x} employee and managers
  , we can get that using the set intersection as shown at line {\tt S9}. Line 
  {\tt S8}, shows the multiline comment we support. Line {\tt S12} shows a way to 
  iterate over the set elements using foreach construct.

  Again, if we want to get the  set of {\tt \_\_DPT\_x} employees who are 
  \textbf{NOT} {\tt \_\_MGR},
  we can write that query as in line {\tt S17}. Line {\tt S19} shows another way
    to get an iterate on composite data structures. 

 Now once we have two sets of nodes each with a specific property, we can query 
 for the edges between them.
 For example, lines {\tt S28}, {\tt S31} and {\tt S34} gives the set of edges 
   emanating from one set of nodes
 to other set of nodes. For example, {\tt Set1 $\rightarrow$ Set2} gives all the 
 directed edges from vertices in set {\tt Set1} to the vertices in {\tt Set 2}. 
 Similarly, {\tt Set1 $\leftarrow$ Set2} gives all the directed edges from 
 vertices in set {\tt Set2} to the vertices in {\tt Set 1}.
  Note that operators $\leftarrow$ and $\rightarrow$ are 
 applicable to directed graphs
 and the operator $\leftrightarrow$ is applicable to both directed and 
 undirected graphs. For directed graphs,
 it gives all the bidirectional edges between two node sets and for undirected 
   graphs, it returns all the edges between
   two node sets.
    \hfill\psframebox{}}
\end{example}

\begin{example} {\rm
  Figure~\ref{fig:motiv_5} represents another example implementation
    the depth first traversal on a graph in our proposed language. 
    \begin{figure}
\begin{center}
      {\small \tt
        \begin{tabular}[b]{rl}
          {\em \scriptsize S1:}&  define("NUM\_VERTICES", "10");\\
          {\em \scriptsize S2:}&  define("NOTVISITED", "0");\\
          {\em \scriptsize S3:}&  define("VISITED", "1"); \\
          &function dfs(v, dfsorder) \{ \\
          {\em \scriptsize S4:}&  \quad  if(v.visit == VISITED)\\
          {\em \scriptsize S5:}&  \quad  \quad return;\\
          {\em \scriptsize S6:}&  \quad   \\
          {\em \scriptsize S7:}&  \quad  println("vertex visited: " + v.num);\\
          {\em \scriptsize S8:}&  \quad  v.visit = VISITED;\\
          {\em \scriptsize S9:}&  \quad   \\
          {\em \scriptsize S10:}&  \quad  dfsorder.pushBack(v); \\
          {\em \scriptsize S11:}&  \quad   \\
          {\em \scriptsize S12:}&  \quad foreach(neighbor; v.getNeighbors())\\
          {\em \scriptsize S13:}&  \quad \quad dfs(neighbor, dfsorder); \\
          &\} \\
          &   \\
          &function main(argv) \{ \\
          {\em \scriptsize S14:}&  \quad  g = graph(); \\
          {\em \scriptsize S15:}&  \quad  g.loadFromFile(argv[0]);\\
          {\em \scriptsize S16:}&  \quad   \\
          {\em \scriptsize S17:}&  \quad  dfsorder = array(0);\\
          {\em \scriptsize S18:}&  \quad  dfs(first, dfsorder);\\
          {\em \scriptsize S19:}&  \quad   \\
          {\em \scriptsize S20:}&  \quad  for(vertex: dfsorder ) \{ \\
          {\em \scriptsize S21:}&  \quad \quad println(" " + vertex.\_\_id + " " 
              ); \\
          {\em \scriptsize S22:}& \quad \}\\
          &\}
        \end{tabular}
      }
\end{center}
  \caption{Example code snippet for dfs traversal on a graph}
  \label{fig:motiv_5}
\end{figure}

  In this example we are trying to store the dfs traversal order of the vertices 
    as well. For that we declared an array at line {\tt S17} and used it to 
    store the traversal order at line {\tt S10}. The method {\tt getNeighbors()} 
  at line {\tt S12} is again a built-in function and returns the neighboring 
    vertices of the receiver vertex. 
    \hfill\psframebox{}}
\end{example}
